\documentclass[12pt,a4paper]{article}
%\usepackage{epsf,epic,eepic,eepicemu}
%\documentstyle[epsf,epic,eepic,eepicemu]{article}

\usepackage[pdftex]{graphicx}
\usepackage[utf8]{inputenc} %kodovani znaku v textovem souboru
%\usepackage[T1]{fontenc} %kodovani znaku na vystupu
\usepackage[czech]{babel} %prizpusobeni jazyku, napr. deleni slov
%\usepackage{a4wide}

%\usepackage[margin=2.5cm]{geometry}

\usepackage{url}

\usepackage{setspace}
\onehalfspacing


\begin{document}
\title{Semestrální projekt MI-PAP, MI-PRC 2014/2015\\
Násobení matic \\
\vspace{10px}}
\author{Karel Fiala \\
\vspace{10px} \\
\small České vysoké učení technické v~Praze\\
\small Fakulta informačních technologií\\
\small Thákurova 9, 160 00 Praha 6\\
\small Česká republika \\
\vspace{10px} \\
}
\date{\today}
\maketitle
\thispagestyle{empty}

% takhle se pise komentar

\clearpage
\thispagestyle{empty}
\tableofcontents
\clearpage

%\begin{abstract}
%
%Your abstract goes here...
%...
%\end{abstract}
%\clearpage


\part*{část jedna}						% * zaridi ze to nebude cislovane
\addcontentsline{toc}{part}{Úvod}  % toto přidá popisek do obsahu
%doplnit text
\pagebreak


\section{sekce 1}
\subsection{subsekce 1}


\begin{itemize}
\item seznam
\end{itemize}


%% tabulka
%\begin{center}
%\begin{tabular}{ | c || c | c | c | c | }
%\hline
%N    &   čas	\\
%\hline
%\hline
%1000    &   0.20681 	\\ \hline
%2000    &   0.39094 	\\ \hline
%3000    &   0.82996 	\\ \hline
%4000    &   1.27902 	\\ \hline
%5000    &   2.92957 	\\ \hline
%6000    &   4.28045 	\\ \hline
%7000    &   7.42563 	\\ \hline
%8000    &   8.21492 	\\ \hline
%\end{tabular}
%\end{center}
%
%
%% obrázek
%\pagebreak
%\begin{figure}[h]
%\includegraphics[width=\textwidth]{graph/cuda/cuda1.png}
%\caption{Klasický multiplikativní algoritmus -- CUDA implementace}
%\label{cuda1}
%\end{figure}

% generujeme literaturu
% PRŮCHA, Jan a Soňa KOŤÁTKOVÁ. Předškolní pedagogika: učebnice pro střední a vyšší odborné školy. Vyd. 1. Praha: Portál, 2013, 181 s. ISBN 978-80-262-0495-4.
%\bibliographystyle{iso690}
%\bibliography{literatura}

%\appendix


\end{document}
