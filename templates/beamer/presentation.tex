\documentclass{beamer}
%\usetheme{PaloAlto}
\usetheme{Boadilla}
\usecolortheme{crane}
%\mode<presentation>

\usepackage[utf8]{inputenc}
\usepackage[czech]{babel}

\usepackage{graphicx}
\usepackage{caption}
\usepackage{subcaption}

\usepackage{pifont} % symboly \ding{}


\makeatletter
\def\Ldescription{%
  \@ifnextchar[{\beamer@testforospec}{\beamer@descdefault\beamer@descriptionwidth\@@Ldescription}%
}

\def\beamer@testforospec[{\@ifnextchar<{\beamer@scandefaultospec[}{\@Ldescription[}}%

\def\beamer@scandefaultospec[#1]{\def\beamer@defaultospec{#1}\Ldescription}

\def\@Ldescription[#1]{%
\setbox\beamer@tempbox=\hbox{\def\insertdescriptionitem{#1}
  \usebeamertemplate**{description item}}%
\beamer@descdefault\wd\beamer@tempbox\@@description%
}%

\def\@@Ldescription{%
  \beamer@descdefault35pt%
  \list
  {}
  {\labelwidth\beamer@descdefault\leftmargin2.8em\let\makelabel\beamer@Ldescriptionitem}%
  \beamer@cramped%
  \raggedright
  \beamer@firstlineitemizeunskip%
}

\def\endLdescription{\ifhmode\unskip\fi\endlist}
\long\def\beamer@Ldescriptionitem#1{%
  \def\insertdescriptionitem{#1}%
  \hspace\labelsep{\parbox[b]{\dimexpr\textwidth-\labelsep\relax}{%
        \usebeamertemplate**{description item}%
    }}}
\makeatother

%==============================================================================



\title[BI-PPR - Projekt, prezentace, rétorika]{Prezentace bakalářské práce}
\subtitle[BI-PPR - Projekt, prezentace, rétorika]{Téma: Individuální třídní schůzky}
\author[Karel Fiala]{Karel Fiala}
\institute[ČVUT FIT]{
   Fakulta informačních technologií \\
   České vysoké učení technické v~Praze
 }
\date{\today}


\begin{document}

\begin{frame}
   \titlepage
\end{frame}


\begin{frame}{Obsah prezentace}
   \tableofcontents
\end{frame}


%\section{Úvodem}\label{sec:uvod}
\section{Individuální třídní schůzky}
\begin{frame}{Individuální třídní schůzky}
Individuální třídní schůzky jsou trendem poslední doby. Snaha o zefektivnění vzdělávání žáků pomocí individuálního přístupu. \break
  \begin{itemize}
  \item Co to znamená? \break
  \item K čemu je to dobré? \break
  \item Jak to funguje nyní? \break
  \item Jak to bude fungovat poté? \break
  \end{itemize}
\end{frame}


\section{Představení bakalářské práce}
\begin{frame}{Představení bakalářské práce}

Informační systém pro správu, evidenci a organizaci individuálních třídních schůzek dále ITS \\ 
\begin{flushleft}
  Jak to bude fungovat:\\
    \begin{itemize}
    \item Vedení školy vypíše období, ve kterém probíhají ITS
    \item Učitelé vypíší termíny, které jim časově vyhovují
    \item Rodiče se přihlásí na konkrétní termín
    \item Proběhne schůzka mezi učitelem, rodiče a žákem
    \item \ldots bude to tak jednoduché?
    \end{itemize} 
\end{flushleft}

Vedoucí: RNDr. Helena Wallenfelsová \\
Oponent: Bc. Ing Ivan Ryant
\end{frame}


\section{Cíl bakalářské práce}
\begin{frame}{Cíl bakalářské práce}
Implementovat informační systém pro správu, evidenci a organizaci individuálních třídních schůzek, který bude:
\break
  \begin{itemize}
  \item jednoduchý \\ - každému jen to, co nutně potřebuje \break
  \item přehledný \\ - a také intuitivní \break
  \item robustní \\ - omyly uživatelé, úmyslné pokusy způsobit chybu \break
  \item efektivní \\ - cílem je vše usnadnit nikoliv zkomplikovat \break
  \end{itemize}
\end{frame}


\section{Aktuální stav řešení}
\begin{frame}{Aktuální stav řešení}
  \begin{itemize}
    \item Vymyslet téma \surd \\
    \item Sepsat zadání \surd \\
    \item Sehnat vedoucího \surd \\
    \item Sehnat oponenta \surd \\
    \item Počkat na schválení !!!
    \item Analyzovat požadavky \surd \\
    \item Vytvořit schémata a modely
    \item Implementovat IS
    \item Otestovat
    \item Sepsat dokumentaci
  \end{itemize}
\end{frame}


\section{Dotazy}
\begin{frame}{Dotazy}
  \begin{center}
    Prostor pro Vaše dotazy
    \break \break \break
    Děkuji za pozornost \\
  \end{center}
\end{frame}


%\section{Závěr}\label{sec:zaver}
%\subsection{Shrnutí}\label{sec:zaver1}
%\subsection{Literatura}\label{sec:zaver2}
%\subsection{Poděkování}\label{sec:zaver3}
%\subsection{Otázky}\label{sec:zaver4}




%\begin{frame}{obr}
 %   \begin{figure}[h!]
%	    \centering
%	    \includegraphics[width=0.9\textwidth]{nic.jpg}
%	    \caption{obr}
%	    \label{obr:obr}
 %   \end{figure}
%\end{frame}


\end{document}
